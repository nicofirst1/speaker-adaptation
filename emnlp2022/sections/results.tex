% !TEX root = ../main.tex

%==========================================
\section{Results}
\label{sec:results}
%==========================================

\note{1 page}\\
\note{Ece + Nico}

\raq{the key results are about the listener; the speaker results can be in the appendix}

\subsection{Listeners}
Here, we provide the results yielded by the set of listeners we train. The metrics we report are reference resolution accuracy and mean reciprocal rank. We consider the performance of domain-specific listeners on in-domain and out-domain data separately. 

\subsection{Speakers}
Here, we \ece{briefly} report the performance of the generic speaker and the finetuned domain-specific speakers in terms of NLG metrics such as BLEU, ROGUE, CIDEr and BERTScore \ece{cite}. 

\ece{For more details, see Appendix ...}
In addition, we feed the generated utterances to the best listener models to obtain reference resolution accuracy and mean reciprocal rank. We report the differences in accuracy we observe when we feed the gold utterances vs. speaker-generated utterances into the listeners.

\subsection{Simulators}
We obtain the resolution accuracies of the simulators and plot their capacity to predict listener behavior \ece{gold vs. generated utterances?}. 

\subsection{Adaptive Speaker}

\ece{this paragraph will mention the ones we actually implement, some are more relevant to analyses than results}\textit{To measure the success of adaptation, we look at whether the speaker's utterances are sufficient enough to predict the targets correctly / approximate the prediction distribution of the listener / cause a change in the distribution towards a more correct distribution (images ranked based on similarity to target) / the proportion of CLIP-distilled discriminator words in the adapted sentences / in-domain and out-domain adaptation ...}
