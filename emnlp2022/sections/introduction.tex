% !TEX root = ../main.tex

%==========================================
\section{Introduction}
\label{sec:intro}
%==========================================

\note{Abstract + Introduction: max 1.5 pages}\\
\note{Ece + Mario}

\begin{itemize}
\item To communicate effectively, speaker and listener co-ordinate their use and interpretation of language, within the context of a particular exchange \cite{GarrodAnderson1987}.  In a conversation, speakers develop a language that is specific to the state of affairs and relies on dynamically / interactively established "conceptual pacts" \cite{GarrodAnderson1987,BrennanClark1996}. For example, a speaker's subsequent references \cite{mcdonald-1978-subsequent-reference} to the same entities in a conversation become attuned to the descriptions and interpretation of their conversation partners. References tend to become partner-specific \cite{BrennanClark1996,metzing2003conceptual,brennan2009partner}---because speakers reuse expressions that were successfully interpreted by their immediate partner---and more efficient---as common ground builds up between the interlocutors \cite{stalnaker2002common}, parts of referring expressions can be left implicit \cite{Grice75,ClarkWilkes-Gibbs1986,clark1991grounding,Clark1996}. 

\item Models of reference that consider the conversational history between interlocutors offer a better account of experimental human data \cite{BrennanClark1996,hawkins2020characterizing} and they are more effective for the generation of human-like referring expressions in conversation \cite{takmaz-etal-2020-refer,hawkins2020continual}. This line of work is based on the idea that the general conventions of meaning serve as starting points for interpretation, and may be overwritten by more local and ad hoc conventions set up during the course of a dialogue \cite{GarrodAnderson1987,ClarkWilkes-Gibbs1986}

\item But what if general conventions are themselves not aligned? What if speaker and listener have access to a different general semantic knowledge?  The listener's knowledge may not be as complete as that of the speaker, as it is the case in adult-child and teacher-learner interactions [cite]. More generally, speakers want to be understood by different conversational partners regardless of their exact store of general semantic knowledge and even though their partners are new to the context of the interaction [example, cite].

\item To coordinate not just at the level of conversation-specific expressions, but also at the level of general semantic knowledge, it is fundamental to be able to represent and reason about others’ mental states \cite{premack1978tom}. Speakers use pragmatic reasoning to make predictions about listeners' interpretation and to adapt their language use accordingly [cite; RSA]. \citet{corona2019modeling} have proposed a machine learning model that forms an internal  representation of other agents that encodes how well they would understand different referring utterances presented to them. They use an asymmetric setup where a proficient speaker learns to adapt to a population of listeners with a different understanding of visual attributes. They show that a mental model over other agents’ understanding of visual attributes makes the interactions more successful.

\item We extend this work by using an image reference game that elicits complex referring expressions---drawing from a full vocabulary rather than from a small subset thereof and without severe length constraints---and by creating a population of listeners that differ according to their general semantic knowledge---as represented by a high-dimensional semantic space---rather than to a set of predefined attributes.

\item We explore whether a proficient speaker can adapt dynamically to different interlocutors by forming a model of the interlocutors' semantic knowledge and using it to generate referring expressions that are both partner- and conversation-specific.

inspired by the plug-and-play approach 


\end{itemize}