% !TEX root = ../main.tex

%==========================================
\section{Introduction}
\label{sec:intro}
%==========================================

\note{Abstract + Introduction: max 1.5 pages}\\
\note{Ece + Mario}
\mar{Still very drafty and incomplete.}

Communication is more effective when a speaker is able to adapt their language to the language of their interlocutors. When adults speak with children, for example, they use simplified expressions to ensure children are able to understand; when computational linguists give a talk at a cognitive science conference, they tend to avoid making extensive use of NLP jargon, as that will prevent their audience from following through the presentation.
The speaker and the listener's language can vary more or less severely; but they are never identical.
Still, speakers are able to communicate with different conversational partners---regardless of their exact store of general semantic knowledge or of whether they are new to the context or topic of the interaction.
Successful adaptation to the semantic knowledge of conversational partners requires the ability to represent and reason about others’ mental states \cite{tomasello2005constructing}. This social-cognitive ability is often referred to as Theory of Mind \cite[ToM;][]{premack1978tom}. 

In this paper, we model an asymmetric multi-agent communication scenario in which a proficient speaker interacts with listeners characterised by limited semantic knowledge in order to successfully complete a reference game.
In a reference game, the goal is for participants to produce descriptions that allow comprehenders to identify an entity in context. These games have been extensively used to study human strategies for effective reference, with speakers produced expressions to differentiate cards with ambiguous figures and coloured chips \cite{krauss1964changes,krauss1967effect}, as well as geometric figures, tangrams, and everyday objects cut from catalogues \citep{ClarkWilkes-Gibbs1986,BrennanClark1996}. More recently, the release of massive datasets of real visual scenes and the advent of crowdsourced experiments have favoured the collection of large scale reference game datasets \citep{shore-etal-2018-kth,haber2019photobook}, which allow referring expression production to be modelled with modern methods of statistical learning. 

We focus on Referring Expression Generation~\citep[REG;][]{reiter1997building,krahmer-van-deemter-2012-computational} in multimodal dialogues and use REG models equipped with a visual module to generate discriminative image descriptions within a set of related images \cite{andreas-klein-2016-reasoning,vedantam2017context,zarriess-schlangen-2019-know}. We provide our generation model with a ToM module that allows it to form a representation of listener's mental states. The ToM module can be used in a plug-and-play fashion \cite{dathathri2020plug}: it preserves the knowledge of the proficient speaker while making it possible to tailor image descriptions to the semantic knowledge of the listener.

We show...

\begin{comment}
\begin{itemize}
\item To communicate effectively, speaker and listener co-ordinate their use and interpretation of language, within the context of a particular exchange \cite{GarrodAnderson1987}.  In a conversation, speakers develop a language that is specific to the state of affairs and relies on dynamically / interactively established "conceptual pacts" \cite{GarrodAnderson1987,BrennanClark1996}. For example, a speaker's subsequent references \cite{mcdonald-1978-subsequent-reference} to the same entities in a conversation become attuned to the descriptions and interpretation of their conversation partners. References tend to become partner-specific \cite{BrennanClark1996,metzing2003conceptual,brennan2009partner}---because speakers reuse expressions that were successfully interpreted by their immediate partner---and more efficient---as common ground builds up between the interlocutors \cite{stalnaker2002common}, parts of referring expressions can be left implicit \cite{Grice75,ClarkWilkes-Gibbs1986,clark1991grounding,Clark1996}. 

\item Models of reference that consider the conversational history between interlocutors offer a better account of experimental human data \cite{BrennanClark1996,hawkins2020characterizing} and they are more effective for the generation of human-like referring expressions in conversation \cite{takmaz-etal-2020-refer,hawkins2020continual}. This line of work is based on the idea that the general conventions of meaning serve as starting points for interpretation, and may be overwritten by more local and ad hoc conventions set up during the course of a dialogue \cite{GarrodAnderson1987,ClarkWilkes-Gibbs1986}

\item But what if general conventions are themselves not aligned? What if speaker and listener have access to a different general semantic knowledge?  The listener's knowledge may not be as complete as that of the speaker, as it is the case in adult-child and teacher-learner interactions [cite]. More generally, speakers want to be understood by different conversational partners regardless of their exact store of general semantic knowledge and even though their partners are new to the context of the interaction [example, cite].

\item To coordinate not just at the level of conversation-specific expressions, but also at the level of general semantic knowledge, it is fundamental to be able to represent and reason about others’ mental states \cite{premack1978tom}. Speakers use pragmatic reasoning to make predictions about listeners' interpretation and to adapt their language use accordingly [cite; RSA]. \citet{corona2019modeling} have proposed a machine learning model that forms an internal  representation of other agents that encodes how well they would understand different referring utterances presented to them. They use an asymmetric setup where a proficient speaker learns to adapt to a population of listeners with a different understanding of visual attributes. They show that a mental model over other agents’ understanding of visual attributes makes the interactions more successful.

\item We extend this work by using an image reference game that elicits complex referring expressions---drawing from a full vocabulary rather than from a small subset thereof and without severe length constraints---and by creating a population of listeners that differ according to their general semantic knowledge---as represented by a high-dimensional semantic space---rather than to a set of predefined attributes.

\item Inspired by 
% [IS THIS EVEN TRUE?] Wittgenstein's language games \citep{wittgenstein1953philosophical} and 
Lewis' signaling games \citep{lewis1969convention}, \textit{reference games} of various forms have been used in linguistics as an experimental setup to study human language production strategies. In a reference game, the goal is for participants to produce descriptions that allow comprehenders to identify an entity in context. In early experiments, speakers produced expressions to differentiate cards with ambiguous figures and coloured chips \cite{krauss1964changes,krauss1967effect}, as well as geometric figures, tangrams, and everyday objects cut from catalogues \citep{ClarkWilkes-Gibbs1986,BrennanClark1996}. More recently, the release of massive datasets of real visual scenes and the advent of crowdsourced experiments have favoured the collection of large scale reference game datasets \citep{shore-etal-2018-kth,haber2019photobook}, which allow referring expression production to be modelled with modern methods of statistical learning as we do in this paper.

\item The production of referring expressions has been studied extensively also in computational linguistics, principally in the branch of Natural Language Generation, and often under the name of Referring Expression Generation~\citep[REG;][]{reiter1997building,krahmer-van-deemter-2012-computational}. The goal of REG research is to build computational models that explain and reproduce human strategies for effective reference in context. 
In this paper, we focus on REG in multimodal dialogues. We therefore use REG models equipped with a visual module, which generate discriminative image descriptions within a set of related images \cite{andreas-klein-2016-reasoning,vedantam2017context,zarriess-schlangen-2019-know}. Our REG models are also conditioned on the conversational history between interlocutors.
History-aware models of reference offer a better account of experimental human data \cite{BrennanClark1996,hawkins2020characterizing} and they lead to human-like referring expressions \cite{dusek-jurcicek-2016-context,hawkins2020continual,takmaz-etal-2020-refer}. 

\item We explore whether a proficient speaker can adapt dynamically to different interlocutors by forming a model of the interlocutors' semantic knowledge and using it to generate referring expressions that are both partner- and conversation-specific.

inspired by the plug-and-play approach 


\end{itemize}
\end{comment}