% This must be in the first 5 lines to tell arXiv to use pdfLaTeX, which is strongly recommended.
\pdfoutput=1
% In particular, the hyperref package requires pdfLaTeX in order to break URLs across lines.

\documentclass[11pt]{article}

% Remove the "review" option to generate the final version.
\usepackage[review]{emnlp2022}

% Standard package includes
\usepackage{times}
\usepackage{latexsym}

% For proper rendering and hyphenation of words containing Latin characters (including in bib files)
\usepackage[T1]{fontenc}
% For Vietnamese characters
% \usepackage[T5]{fontenc}
% See https://www.latex-project.org/help/documentation/encguide.pdf for other character sets

% This assumes your files are encoded as UTF8
\usepackage[utf8]{inputenc}

% This is not strictly necessary, and may be commented out.
% However, it will improve the layout of the manuscript,
% and will typically save some space.
\usepackage{microtype}

% This is also not strictly necessary, and may be commented out.
% However, it will improve the aesthetics of text in
% the typewriter font.
%\usepackage{inconsolata}


\newcommand{\raq}[1]{\textcolor{blue}{[R: #1]}}
\newcommand{\san}[1]{\textcolor{orange}{[S: #1]}}
\newcommand{\mar}[1]{\textcolor{olive}{[M: #1]}}
\newcommand{\ece}[1]{\textcolor{violet}{[E: #1]}}
\newcommand{\nic}[1]{\textcolor{magenta}{[J: #1]}}

\newcommand{\note}[1]{\textcolor{red}{[#1]}}



% If the title and author information does not fit in the area allocated, uncomment the following
%
%\setlength\titlebox{<dim>}
%
% and set <dim> to something 5cm or larger.

\title{Speaker Adaptation to Listener Domains in Multimodal Dialogue \note{think about the title}}

% Author information can be set in various styles:
% For several authors from the same institution:
% \author{Author 1 \and ... \and Author n \\
%         Address line \\ ... \\ Address line}
% if the names do not fit well on one line use
%         Author 1 \\ {\bf Author 2} \\ ... \\ {\bf Author n} \\
% For authors from different institutions:
% \author{Author 1 \\ Address line \\  ... \\ Address line
%         \And  ... \And
%         Author n \\ Address line \\ ... \\ Address line}
% To start a seperate ``row'' of authors use \AND, as in
% \author{Author 1 \\ Address line \\  ... \\ Address line
%         \AND
%         Author 2 \\ Address line \\ ... \\ Address line \And
%         Author 3 \\ Address line \\ ... \\ Address line}

\author{First Author \\
  Affiliation / Address line 1 \\
  Affiliation / Address line 2 \\
  Affiliation / Address line 3 \\
  \texttt{email@domain} \\\And
  Second Author \\
  Affiliation / Address line 1 \\
  Affiliation / Address line 2 \\
  Affiliation / Address line 3 \\
  \texttt{email@domain} \\}

\begin{document}
\maketitle
\begin{abstract}
\note{start with the general problem we tackle: speakers do not have the same experiences/knowledge, yet we are able to communicate by adapting on the fly in conversation ... presumably this is due to our ToM capability... In this paper we investigate this issues in the context of a referential task...  } 
\raq{be careful with saying the work is on dialogue, because we do not use the whole dialogue and reviewers may complain}


Agents exposed to different visual or linguistic domains may have varying proficiency across domains. When referring to visual content coming from a certain domain, we might adapt the way we describe them in line with our impression of our interlocutors' levels of comprehension in that domain. In our work, inspired by the plug-and-play approach to language model adaptation, we propose models of speaker adaptation to listener domains with the aim of improving listeners' comprehension and task success.

 We use a dataset of multimodal dialogues to train domain-specific listeners and an all-knowing speaker. We then provide the speaker with the ability to adapt to listeners, simulating their behavior. Since a speaker would not have direct access to the internal states of their interlocutors, we exploit listener behavior to train the speaker's theory of mind module. In this way, the speaker is able to simulate the listeners and modify its outputs to obtain referring utterances tailored more to a listener that might lack information on out-domain concepts (linguistic and visual) ...

\end{abstract}



% !TEX root = ../main.tex

%==========================================
\section{Introduction}
\label{sec:intro}
%==========================================

\note{Abstract + Introduction: max 1.5 pages}\\
\note{Ece + Mario}
\mar{Still very drafty and incomplete.}

Communication is more effective when a speaker is able to adapt their language to the language of their interlocutors. When adults speak with children, for example, they use simplified expressions to ensure children are able to understand; when computational linguists give a talk at a cognitive science conference, they tend to avoid making extensive use of NLP jargon, as that will prevent their audience from following through the presentation.
The speaker and the listener's language can vary more or less severely; but they are never identical.
Still, speakers are able to communicate with different conversational partners---regardless of their exact store of general semantic knowledge or of whether they are new to the context or topic of the interaction.
Successful adaptation to the semantic knowledge of conversational partners requires the ability to represent and reason about others’ mental states \cite{tomasello2005constructing}. This social-cognitive ability is often referred to as Theory of Mind \cite[ToM;][]{premack1978tom}. 

In this paper, we model an asymmetric multi-agent communication scenario in which a proficient speaker interacts with listeners characterised by limited semantic knowledge in order to successfully complete a reference game.
In a reference game, the goal is for participants to produce descriptions that allow comprehenders to identify an entity in context. These games have been extensively used to study human strategies for effective reference, with speakers produced expressions to differentiate cards with ambiguous figures and coloured chips \cite{krauss1964changes,krauss1967effect}, as well as geometric figures, tangrams, and everyday objects cut from catalogues \citep{ClarkWilkes-Gibbs1986,BrennanClark1996}. More recently, the release of massive datasets of real visual scenes and the advent of crowdsourced experiments have favoured the collection of large scale reference game datasets \citep{shore-etal-2018-kth,haber2019photobook}, which allow referring expression production to be modelled with modern methods of statistical learning. 

We focus on Referring Expression Generation~\citep[REG;][]{reiter1997building,krahmer-van-deemter-2012-computational} in multimodal dialogues and use REG models equipped with a visual module to generate discriminative image descriptions within a set of related images \cite{andreas-klein-2016-reasoning,vedantam2017context,zarriess-schlangen-2019-know}. We provide our generation model with a ToM module that allows it to form a representation of listener's mental states. The ToM module can be used in a plug-and-play fashion \cite{dathathri2020plug}: it preserves the knowledge of the proficient speaker while making it possible to tailor image descriptions to the semantic knowledge of the listener.

We show...

\begin{comment}
\begin{itemize}
\item To communicate effectively, speaker and listener co-ordinate their use and interpretation of language, within the context of a particular exchange \cite{GarrodAnderson1987}.  In a conversation, speakers develop a language that is specific to the state of affairs and relies on dynamically / interactively established "conceptual pacts" \cite{GarrodAnderson1987,BrennanClark1996}. For example, a speaker's subsequent references \cite{mcdonald-1978-subsequent-reference} to the same entities in a conversation become attuned to the descriptions and interpretation of their conversation partners. References tend to become partner-specific \cite{BrennanClark1996,metzing2003conceptual,brennan2009partner}---because speakers reuse expressions that were successfully interpreted by their immediate partner---and more efficient---as common ground builds up between the interlocutors \cite{stalnaker2002common}, parts of referring expressions can be left implicit \cite{Grice75,ClarkWilkes-Gibbs1986,clark1991grounding,Clark1996}. 

\item Models of reference that consider the conversational history between interlocutors offer a better account of experimental human data \cite{BrennanClark1996,hawkins2020characterizing} and they are more effective for the generation of human-like referring expressions in conversation \cite{takmaz-etal-2020-refer,hawkins2020continual}. This line of work is based on the idea that the general conventions of meaning serve as starting points for interpretation, and may be overwritten by more local and ad hoc conventions set up during the course of a dialogue \cite{GarrodAnderson1987,ClarkWilkes-Gibbs1986}

\item But what if general conventions are themselves not aligned? What if speaker and listener have access to a different general semantic knowledge?  The listener's knowledge may not be as complete as that of the speaker, as it is the case in adult-child and teacher-learner interactions [cite]. More generally, speakers want to be understood by different conversational partners regardless of their exact store of general semantic knowledge and even though their partners are new to the context of the interaction [example, cite].

\item To coordinate not just at the level of conversation-specific expressions, but also at the level of general semantic knowledge, it is fundamental to be able to represent and reason about others’ mental states \cite{premack1978tom}. Speakers use pragmatic reasoning to make predictions about listeners' interpretation and to adapt their language use accordingly [cite; RSA]. \citet{corona2019modeling} have proposed a machine learning model that forms an internal  representation of other agents that encodes how well they would understand different referring utterances presented to them. They use an asymmetric setup where a proficient speaker learns to adapt to a population of listeners with a different understanding of visual attributes. They show that a mental model over other agents’ understanding of visual attributes makes the interactions more successful.

\item We extend this work by using an image reference game that elicits complex referring expressions---drawing from a full vocabulary rather than from a small subset thereof and without severe length constraints---and by creating a population of listeners that differ according to their general semantic knowledge---as represented by a high-dimensional semantic space---rather than to a set of predefined attributes.

\item Inspired by 
% [IS THIS EVEN TRUE?] Wittgenstein's language games \citep{wittgenstein1953philosophical} and 
Lewis' signaling games \citep{lewis1969convention}, \textit{reference games} of various forms have been used in linguistics as an experimental setup to study human language production strategies. In a reference game, the goal is for participants to produce descriptions that allow comprehenders to identify an entity in context. In early experiments, speakers produced expressions to differentiate cards with ambiguous figures and coloured chips \cite{krauss1964changes,krauss1967effect}, as well as geometric figures, tangrams, and everyday objects cut from catalogues \citep{ClarkWilkes-Gibbs1986,BrennanClark1996}. More recently, the release of massive datasets of real visual scenes and the advent of crowdsourced experiments have favoured the collection of large scale reference game datasets \citep{shore-etal-2018-kth,haber2019photobook}, which allow referring expression production to be modelled with modern methods of statistical learning as we do in this paper.

\item The production of referring expressions has been studied extensively also in computational linguistics, principally in the branch of Natural Language Generation, and often under the name of Referring Expression Generation~\citep[REG;][]{reiter1997building,krahmer-van-deemter-2012-computational}. The goal of REG research is to build computational models that explain and reproduce human strategies for effective reference in context. 
In this paper, we focus on REG in multimodal dialogues. We therefore use REG models equipped with a visual module, which generate discriminative image descriptions within a set of related images \cite{andreas-klein-2016-reasoning,vedantam2017context,zarriess-schlangen-2019-know}. Our REG models are also conditioned on the conversational history between interlocutors.
History-aware models of reference offer a better account of experimental human data \cite{BrennanClark1996,hawkins2020characterizing} and they lead to human-like referring expressions \cite{dusek-jurcicek-2016-context,hawkins2020continual,takmaz-etal-2020-refer}. 

\item We explore whether a proficient speaker can adapt dynamically to different interlocutors by forming a model of the interlocutors' semantic knowledge and using it to generate referring expressions that are both partner- and conversation-specific.

inspired by the plug-and-play approach 


\end{itemize}
\end{comment}
% !TEX root = ../main.tex

%==========================================
\section{Related Work}
\label{sec:related-work}
%==========================================

\begin{itemize}
\item Subsequent references \cite{gupta2005automatic,jordan2005learning,stoia-etal-2006-noun,viethen-etal-2011-generating}
\item Dialogue history \cite{brockmann2005modelling,buschmeier-etal-2009-alignment,stoyanchev-stent-2009-lexical,lopes2015rule,hu2016entrainment,dusek-jurcicek-2016-context}
\item Controlled generation \cite{nguyen2017plug,dathathri2020plug,keskar2019ctrl,ziegler2019finetuning}
\item RSA / TOM \cite{premack1978tom,corona2019modeling,andreas-klein-2016-reasoning,vedantam2017context,cohn-gordon-etal-2018-pragmatically}
\item Speaker specificity / domain adaptation

PPLM paper
Akata

Reference-Centric Models for Grounded Collaborative Dialogue
Daniel Fried, Justin T. Chiu, Dan Klein

MindCraft: Theory of Mind Modeling for Situated Dialogue in Collaborative Tasks

\end{itemize}



% !TEX root = ../main.tex

%==========================================
\section{Data}
\label{sec:data}
%==========================================

\note{1 column}\\
\note{Mario}

\begin{itemize}
\item The PhotoBook dataset \citep{haber2019photobook} is a collection of task-oriented visually grounded English dialogues between two participants...
For every given PhotoBook dialogue and target image \emph{i}, \citet{takmaz-etal-2020-refer} obtained a reference chain made up of the referring utterances that refer to \emph{i} in the dialogue. The extraction procedure has a precision of 0.86 and a recall of 0.61, and the extracted chains are very similar to the human-annotated ones in terms of chain and utterance length. The dataset is made up of 41,340 referring utterances and 16,525 chains (i.e., there are 16,525 first descriptions and 24,815 subsequent references). The median number of utterances in a chain is 3.
\item The original images are taken from the Microsoft COCO dataset [cite] and belong to 30 different image domains. We cluster the image domains according based on vocabulary vectors constructed by counting word frequencies in the referring utterances targeting images from a given domain. We obtain a set of 5 macro-domains (indoor, outdoor, food, vehicles, appliances), selected so that the domain vocabularies have little overlap. 
For each cluster of image domains, we extract the PhotoBook reference chains produced in the corresponding visual contexts. We then split these into training, validation, and test set. The test set is made up for two thirds of referring utterances targeting images seen in training (\textit{test seen}) and for one third of utterances targeting unseen images (\textit{test unseen}). 
Listeners are trained on the 5 domain-specific datasets. The proficient speaker is trained on the union of the domain-specific datasets (the training set is the union of the 5 domain-specific datasets, the test unseen set is the union of the 5 domain-specific datasets, and so forth).
\item The task is set up as an image reference game. Given a visual context, the speaker produces a referring utterance that describes the target image. The listener selects an image among those in the visual context and is rewarded when it correctly resolves the speaker's reference.
\end{itemize}




% !TEX root = ../main.tex

%==========================================
\section{Problem Formulation}
\label{sec:problem}
%==========================================

\note{1 column max; sections 1-4 must fit within the first 4 pages of the paper}\\
\note{Ece}

The different setups we compare (abstracting away from the specific models we use to implement them), including a diagram. What are the baselines / which ablations we consider, etc.

% !TEX root = ../main.tex

%==========================================
\section{Models}
\label{sec:models}
%==========================================


We construct a speaker model and a listener model. The speaker observes 6 images, knows which one is the target image, and remembers referring expressions previously uttered to describe that image. Given this input, the speaker produces a referring utterance. The listener observes (the same) 6 images, receives the speaker's referring utterance, and remembers the referring expressions used previously for all images. Given this input, the listener selects the most likely image.
We train a proficient speaker model and domain-specific listener models with true---visually contextualised---referring utterances. During training, the speaker model also learns to predict the behaviour of different listeners, and at testing time it can use this ability to infer the semantic knowledge of the listener and to adapt its utterances accordingly. 


We build upon the models presented by \citet{takmaz-etal-2020-refer}. 

%------------------------------------------
\subsection{Speaker}
\label{sec:speaker}
%------------------------------------------

The Speaker generates a referring utterance given (a) the \emph{visual context} in the current game round made up of $6$ images from the perspective of the player who produced the utterance, (b) the \emph{target} among those images, and (c) the \emph{previous co-referring utterances} in the chain (if any). 


We use the ReRef model proposed and implemeted by \citet{takmaz-etal-2020-refer}, which generates a new utterance conditioned on both the visual and the linguistic context. This model simulates a speaker who is able to produce subsequent references to a target image in accordance with what has been established in the conversational common ground \cite{Clark1996,BrennanClark1996}.  

The encoder is a one-layer bidirectional LSTM... The decoder is a unidirectional LSTM... We use nucleus sampling rather than beam-search, with parameters ...



%------------------------------------------
\subsection{Listener}
\label{sec:listener}
%------------------------------------------

The Listeners try to guess the correct image given the Speaker's referring utterance, the visual context, and the previous co-referring utterances.


We use the Listener model proposed and implemeted by \citet{takmaz-etal-2020-refer}. Given an utterance referring to a target image and a $6$-image visual context, this model predicts the target image.

Each candidate image is represented by its ResNet-152 features~\cite{resnet2016}. To pick a referent, we take the dot product between the multimodal representation of the input utterance and each of the candidate image representations. The image with the highest dot-product value is the one chosen by the model. 

%------------------------------------------
\subsection{Adaptive Speaker - PPLM}
\label{sec:adapts}
%------------------------------------------
PPLM in more detail

adaptive speaker trained with the simulator

% !TEX root = ../main.tex

%==========================================
\section{Results}
\label{sec:results}
%==========================================

\note{1 page}\\
\note{Ece + Nico}

\raq{the key results are about the listener; the speaker results can be in the appendix}

\subsection{Listeners}
Here, we provide the results yielded by the set of listeners we train. The metrics we report are reference resolution accuracy and mean reciprocal rank. We consider the performance of domain-specific listeners on in-domain and out-domain data separately. 

\subsection{Speakers}
Here, we \ece{briefly} report the performance of the generic speaker and the finetuned domain-specific speakers in terms of NLG metrics such as BLEU, ROGUE, CIDEr and BERTScore \ece{cite}. 

\ece{For more details, see Appendix ...}
In addition, we feed the generated utterances to the best listener models to obtain reference resolution accuracy and mean reciprocal rank. We report the differences in accuracy we observe when we feed the gold utterances vs. speaker-generated utterances into the listeners.

\subsection{Simulators}
We obtain the resolution accuracies of the simulators and plot their capacity to predict listener behavior \ece{gold vs. generated utterances?}. 

\subsection{Adaptive Speaker}

\ece{this paragraph will mention the ones we actually implement, some are more relevant to analyses than results}\textit{To measure the success of adaptation, we look at whether the speaker's utterances are sufficient enough to predict the targets correctly / approximate the prediction distribution of the listener / cause a change in the distribution towards a more correct distribution (images ranked based on similarity to target) / the proportion of CLIP-distilled discriminator words in the adapted sentences / in-domain and out-domain adaptation ...}

% !TEX root = ../main.tex

%==========================================
\section{Analysis}
\label{sec:analysis}
%==========================================

\note{1.5 pages}\\
\note{Mario}


Possible quantitative analyses / evaluation
\begin{itemize}
\item \textbf{Task-level }evaluation: does the listeners' reference resolution become more accurate as a result of speaker adaptation? We compare listener accuracy and Mean Reciprocal Rank (MRR) in response to adapted utterances with that obtained for utterances generated by a static model \cite{takmaz-etal-2020-refer}.
\item \textbf{Utterance-level}
	\begin{itemize}
	\item We compute several metrics that are commonly used for Natural Language Generation. We consider three measures based on \emph{n-}gram matching---BLEU-2~\cite{Papineni:2002},
%\footnote{BLEU-2, which is based on bigrams, appears to be more informative than BLEU with longer $n$-grams in dialogue response generation \cite{liu-etal-2016-evaluate}}
ROUGE~\cite{Lin2004}, and CIDEr~\cite{cider}---as well as BERTScore F1~\cite{bert-score}, which instead of exact string matches relies on the semantic similarity between tokens. With these measures, we capture the degree of similarity between generated referring utterances and their human counterparts. We expect adapted utterances to be less similar to the human references due to more partner-specific language use (not just transient conceptual pacts but coordination of the entire semantic space).
automatic gen metrics
	\item We also measure the length of the generated utterances to check if the adaptive model reproduces the reduction trend found in humans \cite{haber2019photobook,hawkins2020continual} and in history-aware generation models \cite{takmaz-etal-2020-refer,hawkins2020characterizing}.
	\item The speaker's adaptation may lead to language that is highly effective for the purposes of the image reference task but which does not sound natural, human-like (e.g., repetitions, "heavily" ungrammatical sentences). We will compare the distribution of POS sequences produced by the models with those produced by humans (e.g., we know that Noun Noun sequences become more frequent as we move along a reference chain). We will also use chunking to identify idiosyncratic expressions used by the models and/or by humans.
	\end{itemize}
\item \textbf{Vocabulary-level}
	\begin{itemize}
	\item We will look at the POS distribution over the vocabulary produced by the models and by humans over the entire PhotoBook dataset, as a function of adaptation and of utterance position in the chain.
	\item A possible adaptation strategy for speaker is to reduce its vocabulary to a few very effective words. We check if this is the case by monitoring the vocabulary size and type-token ratio of the models (adaptive and static) and comparing it to humans'.
	\item Another way to assessing how human-like the models' vocabularies are is to check if they follow the Zipfian distribution that we expect from a natural language. 
	\end{itemize}
\item \textbf{Model-level}: We also analyse the effect that adaptation has on the model's internal representations and on its lexical choices.
	\begin{itemize}
	\item What type of internal representations is it more useful to modify? Does adaptation at different representational levels produce different kinds of linguistic output?
	\item How much should the model's hidden representations be modified to strike a good balance between general fluency and partner-specificity? 
	\item How do the model's vocabulary distributions change as a result of adaptation? Do they become less (more) uniform? Do certain successful vocabulary items become more prominent? Are there certain timesteps where the change is stronger?
	\item What effect does adaptation have on the decision making of the model? We can look at the generation samples obtained for adaptive and static models and monitor their overlap as a function of the speed and "depth" of adaptation.
	\end{itemize}
\end{itemize}

Possible qualitative analysis
\begin{itemize}
\item \textbf{Overextension}. Three main types of semantic relations connect conventional and overextended referents of a word \cite{rescorla1980}:
	\begin{itemize}
	\item categorical: linking objects that are close in a taxonomy (e.g., bed referring to a sofa)
	\item analogical: linking objects with shared perceptual properties (e.g., giraffe referring to a lamp)
	\item predicate-based: linking objects that co-occur frequently (e.g., wheels referring to a bike)
	\end{itemize}
Analogical relations are those that can most be exploited by the speaker given a static listener model. Good visual representations should allow the model to find abstract similarities between taxonomically distant and rarely co-occurring objects. 

\end{itemize}
% !TEX root = ../main.tex

%==========================================
\section{Conclusion}
\label{sec:conclusion}
%==========================================
...



\section*{Limitations}
EMNLP 2022 requires all submissions to have a section titled ``Limitations'', for discussing the limitations of the paper as a complement to the discussion of strengths in the main text. This section should occur after the conclusion, but before the references. It will not count towards the page limit.  

The discussion of limitations is mandatory. Papers without a limitation section will be desk-rejected without review.
ARR-reviewed papers that did not include ``Limitations'' section in their prior submission, should submit a PDF with such a section together with their EMNLP 2022 submission.

While we are open to different types of limitations, just mentioning that a set of results have been shown for English only probably does not reflect what we expect. 
Mentioning that the method works mostly for languages with limited morphology, like English, is a much better alternative.
In addition, limitations such as low scalability to long text, the requirement of large GPU resources, or other things that inspire crucial further investigation are welcome.

\section*{Ethics Statement}
Scientific work published at EMNLP 2022 must comply with the \href{https://www.aclweb.org/portal/content/acl-code-ethics}{ACL Ethics Policy}. We encourage all authors to include an explicit ethics statement on the broader impact of the work, or other ethical considerations after the conclusion but before the references. The ethics statement will not count toward the page limit (8 pages for long, 4 pages for short papers).

%\section*{Acknowledgements}


% Entries for the entire Anthology, followed by custom entries
\bibliography{adapt}
%\bibliography{anthology,custom}
\bibliographystyle{acl_natbib}

\appendix

\section{Appendix}
\label{sec:appendix}

This is a section in the appendix.

\end{document}
