% This must be in the first 5 lines to tell arXiv to use pdfLaTeX, which is strongly recommended.
\pdfoutput=1
% In particular, the hyperref package requires pdfLaTeX in order to break URLs across lines.

\documentclass[11pt]{article}

% Remove the "review" option to generate the final version.
\usepackage[review]{emnlp2022}

% Standard package includes
\usepackage{times}
\usepackage{latexsym}

% For proper rendering and hyphenation of words containing Latin characters (including in bib files)
\usepackage[T1]{fontenc}
% For Vietnamese characters
% \usepackage[T5]{fontenc}
% See https://www.latex-project.org/help/documentation/encguide.pdf for other character sets

% This assumes your files are encoded as UTF8
\usepackage[utf8]{inputenc}

% This is not strictly necessary, and may be commented out.
% However, it will improve the layout of the manuscript,
% and will typically save some space.
\usepackage{microtype}

% This is also not strictly necessary, and may be commented out.
% However, it will improve the aesthetics of text in
% the typewriter font.
%\usepackage{inconsolata}

\usepackage{color} % ET
\usepackage{comment} % ET
\usepackage{graphicx} % ET
\usepackage{booktabs} % ET
\usepackage{multirow} % ET
\usepackage{amssymb} % ET

\newcommand{\raq}[1]{\textcolor{blue}{[R: #1]}}
\newcommand{\san}[1]{\textcolor{orange}{[S: #1]}}
\newcommand{\mar}[1]{\textcolor{teal}{[M: #1]}}
\newcommand{\ece}[1]{\textcolor{violet}{[E: #1]}}
\newcommand{\nic}[1]{\textcolor{magenta}{[N: #1]}}

\newcommand{\note}[1]{\textcolor{red}{[#1]}}


% If the title and author information does not fit in the area allocated, uncomment the following
%
%\setlength\titlebox{<dim>}
%
% and set <dim> to something 5cm or larger.

% \title{Speaker Adaptation to Listener Domains in Multimodal Dialogue \note{think about the title}}
% SP
%\title{\emph{Do you have the green wheel?} (put an explicative example here)\\Speaker Adaptation to Domain-Specific Listeners in a Referential Game}
\title{Speaker Adaptation in Visually-Grounded Referential Games\\via Theory of Mind}

% MG: Plug-and-Play Speaker Adaptation in Asymmetric Referential Games
%ET ... + via Listener Simulation/ToM? 
% Plug-and-Play Speaker Adaptation in Visually and Linguistically Asymmetric Referential Games
% sandro's example reminded me of the song 'do you know the muffin man'

% Author information can be set in various styles:
% For several authors from the same institution:
% \author{Author 1 \and ... \and Author n \\
%         Address line \\ ... \\ Address line}
% if the names do not fit well on one line use
%         Author 1 \\ {\bf Author 2} \\ ... \\ {\bf Author n} \\
% For authors from different institutions:
% \author{Author 1 \\ Address line \\  ... \\ Address line
%         \And  ... \And
%         Author n \\ Address line \\ ... \\ Address line}
% To start a seperate ``row'' of authors use \AND, as in
% \author{Author 1 \\ Address line \\  ... \\ Address line
%         \AND
%         Author 2 \\ Address line \\ ... \\ Address line \And
%         Author 3 \\ Address line \\ ... \\ Address line}

\author{First Author \\
  Affiliation / Address line 1 \\
  Affiliation / Address line 2 \\
  Affiliation / Address line 3 \\
  \texttt{email@domain} \\\And
  Second Author \\
  Affiliation / Address line 1 \\
  Affiliation / Address line 2 \\
  Affiliation / Address line 3 \\
  \texttt{email@domain} \\}

\begin{document}
\maketitle
\begin{abstract}

%\san{The incipit of the abstract could be made more incisive. Something like: People engaged in a conversation may have very different knowledge about a particular topic or semantic domain. When this is the case, a speaker needs to be able adapt to their interlocutor to achieve communicative success. Human speakers can align/adapt on the fly presumably thanks to our Theory of Mind abilities, which\dots}

People engaged in a conversation may have varying levels of knowledge about a particular topic. When this is the case, speakers need to be able adapt their utterances by taking their interlocutors into account to achieve communicative success. Human speakers can adapt on the fly presumably thanks to our Theory of Mind capabilities, which
%When people with varying semantic knowledge have a conversation, they may only be able to communicate successfully by adapting their utterances on the fly to better align with their interlocutors. This is possible presumably due to our Theory of Mind capabilities, which 
enable us to consider our interlocutors' levels of comprehension and knowledge. In this paper, we model an adaptive speaker playing an asymmetric referential game with listeners that lack visual and linguistic information about certain domains. Inspired by the plug-and-play approach to controlled language generation, our speaker includes a Theory of Mind module that simulates the listener and steers the adaptation of referring utterances on the fly, without any additional training of the underlying language model. Our experiments and analyses show that this approach to adaptation is effective in asymmetric communicative setups. \san{of course here we may want to say a little more but we can't really do it now}


% When people with varying visual or linguistic capabilities have a conversation, they are able to communicate successfully by adapting their utterances on-the-fly to better align with their interlocutors. This is presumably possible due to our Theory of Mind capabilities that enable us to consider our interlocutors' levels of comprehension and knowledge. In this paper, inspired by the plug-and-play approach to language model adaptation, we propose a pipeline for speaker adaptation with the aim of improving communicative success in a referential game. We use data extracted from a dataset of multimodal dialogues to train a speaker and domain-specific listeners. We exploit external listener behavior to train the speaker's Theory of Mind module simulating listeners' behavior from the perspective of the speaker. We observe that, with the help of this simulator, the speaker is able to yield referring utterances that are tailored more towards its audience that might lack linguistic and visual information on out-of-domain concepts.

% Agents exposed to different visual or linguistic domains may have varying proficiency across domains. When referring to visual content coming from a certain domain, we might adapt the way we describe them in line with our impression of our interlocutors' levels of comprehension in that domain. 



% \note{start with the general problem we tackle: speakers do not have the same experiences/knowledge, yet we are able to communicate by adapting on the fly in conversation ... presumably this is due to our ToM capability... In this paper we investigate this issues in the context of a referential task...  } 
% \raq{be careful with saying the work is on dialogue, because we do not use the whole dialogue and reviewers may complain}


% Agents exposed to different visual or linguistic domains may have varying proficiency across domains. When referring to visual content coming from a certain domain, we might adapt the way we describe them in line with our impression of our interlocutors' levels of comprehension in that domain. In our work, inspired by the plug-and-play approach to language model adaptation, we propose models of speaker adaptation to listener domains with the aim of improving listeners' comprehension and task success.

%  We use a dataset of multimodal dialogues to train domain-specific listeners and an all-knowing speaker. We then provide the speaker with the ability to adapt to listeners, simulating their behavior. Since a speaker would not have direct access to the internal states of their interlocutors, we exploit listener behavior to train the speaker's theory of mind module. In this way, the speaker is able to simulate the listeners and modify its outputs to obtain referring utterances tailored more to a listener that might lack information on out-domain concepts (linguistic and visual) ...

\end{abstract}



% !TEX root = ../main.tex

%==========================================
\section{Introduction}
\label{sec:intro}
%==========================================

\note{Abstract + Introduction: max 1.5 pages}\\
\note{Ece + Mario}
\mar{Still very drafty and incomplete.}

Communication is more effective when a speaker is able to adapt their language to the language of their interlocutors. When adults speak with children, for example, they use simplified expressions to ensure children are able to understand; when computational linguists give a talk at a cognitive science conference, they tend to avoid making extensive use of NLP jargon, as that will prevent their audience from following through the presentation.
The speaker and the listener's language can vary more or less severely; but they are never identical.
Still, speakers are able to communicate with different conversational partners---regardless of their exact store of general semantic knowledge or of whether they are new to the context or topic of the interaction.
Successful adaptation to the semantic knowledge of conversational partners requires the ability to represent and reason about others’ mental states \cite{tomasello2005constructing}. This social-cognitive ability is often referred to as Theory of Mind \cite[ToM;][]{premack1978tom}. 

In this paper, we model an asymmetric multi-agent communication scenario in which a proficient speaker interacts with listeners characterised by limited semantic knowledge in order to successfully complete a reference game.
In a reference game, the goal is for participants to produce descriptions that allow comprehenders to identify an entity in context. These games have been extensively used to study human strategies for effective reference, with speakers produced expressions to differentiate cards with ambiguous figures and coloured chips \cite{krauss1964changes,krauss1967effect}, as well as geometric figures, tangrams, and everyday objects cut from catalogues \citep{ClarkWilkes-Gibbs1986,BrennanClark1996}. More recently, the release of massive datasets of real visual scenes and the advent of crowdsourced experiments have favoured the collection of large scale reference game datasets \citep{shore-etal-2018-kth,haber2019photobook}, which allow referring expression production to be modelled with modern methods of statistical learning. 

We focus on Referring Expression Generation~\citep[REG;][]{reiter1997building,krahmer-van-deemter-2012-computational} in multimodal dialogues and use REG models equipped with a visual module to generate discriminative image descriptions within a set of related images \cite{andreas-klein-2016-reasoning,vedantam2017context,zarriess-schlangen-2019-know}. We provide our generation model with a ToM module that allows it to form a representation of listener's mental states. The ToM module can be used in a plug-and-play fashion \cite{dathathri2020plug}: it preserves the knowledge of the proficient speaker while making it possible to tailor image descriptions to the semantic knowledge of the listener.

We show...

\begin{comment}
\begin{itemize}
\item To communicate effectively, speaker and listener co-ordinate their use and interpretation of language, within the context of a particular exchange \cite{GarrodAnderson1987}.  In a conversation, speakers develop a language that is specific to the state of affairs and relies on dynamically / interactively established "conceptual pacts" \cite{GarrodAnderson1987,BrennanClark1996}. For example, a speaker's subsequent references \cite{mcdonald-1978-subsequent-reference} to the same entities in a conversation become attuned to the descriptions and interpretation of their conversation partners. References tend to become partner-specific \cite{BrennanClark1996,metzing2003conceptual,brennan2009partner}---because speakers reuse expressions that were successfully interpreted by their immediate partner---and more efficient---as common ground builds up between the interlocutors \cite{stalnaker2002common}, parts of referring expressions can be left implicit \cite{Grice75,ClarkWilkes-Gibbs1986,clark1991grounding,Clark1996}. 

\item Models of reference that consider the conversational history between interlocutors offer a better account of experimental human data \cite{BrennanClark1996,hawkins2020characterizing} and they are more effective for the generation of human-like referring expressions in conversation \cite{takmaz-etal-2020-refer,hawkins2020continual}. This line of work is based on the idea that the general conventions of meaning serve as starting points for interpretation, and may be overwritten by more local and ad hoc conventions set up during the course of a dialogue \cite{GarrodAnderson1987,ClarkWilkes-Gibbs1986}

\item But what if general conventions are themselves not aligned? What if speaker and listener have access to a different general semantic knowledge?  The listener's knowledge may not be as complete as that of the speaker, as it is the case in adult-child and teacher-learner interactions [cite]. More generally, speakers want to be understood by different conversational partners regardless of their exact store of general semantic knowledge and even though their partners are new to the context of the interaction [example, cite].

\item To coordinate not just at the level of conversation-specific expressions, but also at the level of general semantic knowledge, it is fundamental to be able to represent and reason about others’ mental states \cite{premack1978tom}. Speakers use pragmatic reasoning to make predictions about listeners' interpretation and to adapt their language use accordingly [cite; RSA]. \citet{corona2019modeling} have proposed a machine learning model that forms an internal  representation of other agents that encodes how well they would understand different referring utterances presented to them. They use an asymmetric setup where a proficient speaker learns to adapt to a population of listeners with a different understanding of visual attributes. They show that a mental model over other agents’ understanding of visual attributes makes the interactions more successful.

\item We extend this work by using an image reference game that elicits complex referring expressions---drawing from a full vocabulary rather than from a small subset thereof and without severe length constraints---and by creating a population of listeners that differ according to their general semantic knowledge---as represented by a high-dimensional semantic space---rather than to a set of predefined attributes.

\item Inspired by 
% [IS THIS EVEN TRUE?] Wittgenstein's language games \citep{wittgenstein1953philosophical} and 
Lewis' signaling games \citep{lewis1969convention}, \textit{reference games} of various forms have been used in linguistics as an experimental setup to study human language production strategies. In a reference game, the goal is for participants to produce descriptions that allow comprehenders to identify an entity in context. In early experiments, speakers produced expressions to differentiate cards with ambiguous figures and coloured chips \cite{krauss1964changes,krauss1967effect}, as well as geometric figures, tangrams, and everyday objects cut from catalogues \citep{ClarkWilkes-Gibbs1986,BrennanClark1996}. More recently, the release of massive datasets of real visual scenes and the advent of crowdsourced experiments have favoured the collection of large scale reference game datasets \citep{shore-etal-2018-kth,haber2019photobook}, which allow referring expression production to be modelled with modern methods of statistical learning as we do in this paper.

\item The production of referring expressions has been studied extensively also in computational linguistics, principally in the branch of Natural Language Generation, and often under the name of Referring Expression Generation~\citep[REG;][]{reiter1997building,krahmer-van-deemter-2012-computational}. The goal of REG research is to build computational models that explain and reproduce human strategies for effective reference in context. 
In this paper, we focus on REG in multimodal dialogues. We therefore use REG models equipped with a visual module, which generate discriminative image descriptions within a set of related images \cite{andreas-klein-2016-reasoning,vedantam2017context,zarriess-schlangen-2019-know}. Our REG models are also conditioned on the conversational history between interlocutors.
History-aware models of reference offer a better account of experimental human data \cite{BrennanClark1996,hawkins2020characterizing} and they lead to human-like referring expressions \cite{dusek-jurcicek-2016-context,hawkins2020continual,takmaz-etal-2020-refer}. 

\item We explore whether a proficient speaker can adapt dynamically to different interlocutors by forming a model of the interlocutors' semantic knowledge and using it to generate referring expressions that are both partner- and conversation-specific.

inspired by the plug-and-play approach 


\end{itemize}
\end{comment}
% !TEX root = ../main.tex

%==========================================
\section{Related Work}
\label{sec:related-work}
%==========================================

\note{max 2 columns}\\
\note{Ece + Mario}

\begin{itemize}
\item Subsequent references \cite{gupta2005automatic,jordan2005learning,stoia-etal-2006-noun,viethen-etal-2011-generating}
\item Dialogue history \cite{brockmann2005modelling,buschmeier-etal-2009-alignment,stoyanchev-stent-2009-lexical,lopes2015rule,hu2016entrainment,dusek-jurcicek-2016-context}
\item Controlled generation \cite{nguyen2017plug,dathathri2020plug,keskar2019ctrl,ziegler2019finetuning}
\item RSA / TOM \cite{premack1978tom,corona2019modeling,andreas-klein-2016-reasoning,vedantam2017context,cohn-gordon-etal-2018-pragmatically}
\item Speaker specificity / domain adaptation

PPLM paper
Akata

Reference-Centric Models for Grounded Collaborative Dialogue
Daniel Fried, Justin T. Chiu, Dan Klein

MindCraft: Theory of Mind Modeling for Situated Dialogue in Collaborative Tasks

\end{itemize}



% !TEX root = ../main.tex

%==========================================
\section{Data}
\label{sec:data}
%==========================================

\begin{itemize}
\item The PhotoBook dataset \citep{haber2019photobook} is a collection of task-oriented visually grounded English dialogues between two participants...
For every given PhotoBook dialogue and target image \emph{i}, \citet{takmaz-etal-2020-refer} obtained a reference chain made up of the referring utterances that refer to \emph{i} in the dialogue. The extraction procedure has a precision of 0.86 and a recall of 0.61, and the extracted chains are very similar to the human-annotated ones in terms of chain and utterance length. The dataset is made up of 41,340 referring utterances and 16,525 chains (i.e., there are 16,525 first descriptions and 24,815 subsequent references). The median number of utterances in a chain is 3.
\item The original images are taken from the Microsoft COCO dataset [cite] and belong to 30 different image domains. We cluster the image domains according based on vocabulary vectors constructed by counting word frequencies in the referring utterances targeting images from a given domain. We obtain a set of 5 macro-domains (indoor, outdoor, food, vehicles, appliances), selected so that the domain vocabularies have little overlap. 
For each cluster of image domains, we extract the PhotoBook reference chains produced in the corresponding visual contexts. We then split these into training, validation, and test set. The test set is made up for two thirds of referring utterances targeting images seen in training (\textit{test seen}) and for one third of utterances targeting unseen images (\textit{test unseen}). 
Listeners are trained on the 5 domain-specific datasets. The proficient speaker is trained on the union of the domain-specific datasets (the training set is the union of the 5 domain-specific datasets, the test unseen set is the union of the 5 domain-specific datasets, and so forth).
\item The task is set up as an image reference game. Given a visual context, the speaker produces a referring utterance that describes the target image. The listener selects an image among those in the visual context and is rewarded when it correctly resolves the speaker's reference.
\end{itemize}




% !TEX root = ../main.tex

%==========================================
\section{Problem Formulation}
\label{sec:problem}
%=========`=================================

%\san{Problem formulation could also be Experimental Setup / General comment: to be useful, this section needs to be intuitive from a human communication point of view. This is what motivates our pipeline and comparisons}

\note{1 column max; sections 1-4 must fit within the first 4 pages of the paper}\\
\note{Ece}


\begin{figure*}[t]
	\centering
	\includegraphics[width=2\columnwidth]{images/adaptation.pdf} %\hfill
	\caption{Adaptive pipeline \ece{images and utterances to be replaced with actual data} \san{Nice diagram! I'm not sure I fully understand the visual input, though: why are there 3 images that are the same?}\ece{just placeholders until we have actual outputs from adaptive speaker}} %\san{we could add two heads or robots to indicate that the generic speaker and the listener are two agents, while the domain-specific simulator is an idea (a bulb, a cloud?)} \mar{Should the purple arrow go from the simulator to the latents?}}
	\label{fig:pipeline}
\end{figure*}


%The different setups we compare (abstracting away from the specific models we use to implement them), including a diagram. What are the baselines / which ablations we consider, etc.

In our experimental setting, a speaker and a domain-specific listener play a referential game over a visual context in an asymmetric setup. That is, the listener lacks visual and linguistic information about domains other than its own area of expertise (e.g., \emph{food}).
In such cases, speakers may need to adapt their utterances to align with the domain of the listeners with the aim of achieving communicative success. %accommodate the listener's comprehension capacity with the aim of succeeding in game. In this work, we
%\san{In our experimental setting, a generic speaker and a domain-specific listener (e.g., a listener that only knows about \emph{food}) play a referential game over a visual context. To achieve communicative success, the speaker needs to adapt to the language domain of the listener. Since only the speaker adapts to the listener, the communicative setup is asymmetric. We} 
We model this behavior in our \textit{adaptive pipeline}. We have one listener per domain that we pair with the speaker in the game. Initially, the speaker produces an utterance for a target image among distractors based on its knowledge encompassing all the domains. However, to be successful, it then needs to adapt to the domain of the listener it is paired with, as the reference resolution performance of the listener is constrained by its domain-specific capabilities.

%We create this setting via first exposing the listeners only to \textit{domain-specific subsets} of the full dataset, which is used in training the speaker. The speaker learns to generate an utterance describing the target image from a point of view that encompasses all the domains. The aim of the listener is to predict the target image based on the speaker's utterance, albeit constrained by its own domain-specific capabilities. Via simulating the possible behavior of the listener, the speaker can alter its utterance on-the-fly. \san{All the above can be sign shortened: We have one listener per domain that we pair with the generic speaker in the game. Initially, the speaker utters a generic utterance, but then needs to adapt to the language of the listeners to be successful.}

\noindent\textbf{Adaptive pipeline} %\san{In human communication, a speaker can form their own idea of the listener as the interaction progresses. Inspired by this idea,}
In human communication, a speaker can form their own idea of the listener as the interaction progresses, and adapt their utterances accordingly. Inspired by this idea, in this pipeline, the speaker is endowed with the ability to adapt to the listener's knowledge space with the help of a domain-specific listener `simulator' (see Fig.~\ref{fig:pipeline}).  As the speaker would not have direct access to the internal states of the listeners, it can only use external indicators to guide its utterances: behavioral data coming from the listeners, i.e. listeners' predictions. In this way, the speaker is enhanced with a capacity for ToM and the ability to update its utterances by considering the listener's possible actions. Inspired by the PPLM method~\cite{dathathri2020plug}, \san{are we really getting inspired by this specific method or, more generally, by any CTG method that keep the LM parameters unaltered and operates at the decoding stage? If so, there are a couple more papers we could cite here} we implement an adaptation scheme in which the adaptation is effective only during an exchange with a single listener. In this way, the speaker can retain its domain-general capabilities. We compare the adaptive pipeline against one with a random listener (random), a non-adaptive speaker (baseline), and a domain-specific speaker (ceiling).

% \noindent\textbf{Baselines} 
% \san{The adaptive speaker is compared against a random speaker (random), a non-adaptive speaker (baseline), and domain-specific speaker (ceiling)}
% We compare the adaptive pipeline to 2 baselines: (1) random baseline (1-out-of-6, 16.67\%) and (2) the non-adaptive pipeline with the speaker and the domain-specific listeners, which would be our stronger baseline. As the ceiling, we evaluate domain-specific speakers that would generate utterances more aligned within a given domain without the need for on-the-fly adaptation.


% In this version of our pipeline, we first train a general speaker and a general listener independently. These general agents would be equipped with knowledge coming from all domains. In addition, using the domain-specific data, we train domain-specific listeners that are only exposed to the visual and linguistic concepts existing within the domain. After the training stage, the speaker and the listeners do not learn (i.e. they are frozen). 
% \ece{also pretraining a general listener using \textbf{generated} data from speaker or just evaluating it?} We store the utterances generated by the best speaker, the corresponding latents and the listeners' behavior given these utterances for the adaptive pipeline.
% 
% !TEX root = ../main.tex

%==========================================
\section{Models}
\label{sec:models}
%==========================================

\note{half a page}\note{Ece}\\
\note{Include Experimental Setup (Nico + Ece), possibly as a separate short section; all together this should not fill up page 5}
%In this section, we first explain the general experimental setup and then go into detail about the implementations of the speaker, listener and simulator models as well as the adaptation scheme.
%------------------------------------------
\subsection{Experimental Setup}
\label{sec:exps}
%------------------------------------------
%Model implementation / details / modifications
%\ece{do we need to go into this much detail regarding the speaker and the listener models? if not, which parts should we cut off?} \san{they can probably be shortened, but I think this level of detail is good!}
We build on the referring utterance generation and resolution models proposed by~\citet{takmaz-etal-2020-refer} with modifications as explained below. %Additionally, we filter the data and separate it into domains. 
We represent the images by their ResNet-152 features~\cite{resnet2016}. %\ece{same as RRR, also add CLIP?}
See Appendix X for the hyperparameters and more details about the training of each model type.

%The speaker observes 6 images, knows which one is the target image, and remembers referring expressions previously uttered to describe that image. Given this input, the speaker produces a referring utterance. The listener observes (the same) 6 images, receives the speaker's referring utterance, and remembers the referring expressions used previously for all images. Given this input, the listener selects the most likely image.
 
%We train a proficient speaker model and domain-specific listener models with true---visually contextualised---referring utterances. During training, the speaker model also learns to predict the behaviour of different listeners, and at testing time it can use this ability to infer the semantic knowledge of the listener and to adapt its utterances accordingly. 



%------------------------------------------
\subsection{Speaker}
\label{sec:speaker}
%------------------------------------------

As our speaker model, we use the ReRef model proposed by \citet{takmaz-etal-2020-refer}, which follows the encoder-decoder architecture. % to capture lexical entrainment. %\ece{common ground, conceptual pacts, memory?} 
%This model generates a new utterance referring to a target image in the visual context, taking into account the previous utterances to the same image in the chain. 
The encoder is a bidirectional LSTM initialized with the representation of the visual context and the target image. If a previous utterance to the target exists in the chain, 
the encoder takes the utterance as input. Otherwise, it takes in a special token. In our model, we only use this special token and do not employ ground truth previous utterances, as they may complicate the adaptation process. The final hidden states of the encoder from both directions initialize the unidirectional LSTM decoder. The decoder generates the next utterance via beam search. We train the model from scratch using our dataset and optimize the model with respect to Cross Entropy Loss with the Adam optimizer. We select the best model based on BERTScore F1 on the validation set. 

%This model simulates a speaker who is able to produce subsequent references to a target image in accordance with what has been established in the conversational common ground \cite{Clark1996,BrennanClark1996}.  
%The Speaker generates a referring utterance given (a) the \emph{visual context} in the current game round made up of $6$ images from the perspective of the player who produced the utterance, (b) the \emph{target} among those images, and (c) the \emph{previous co-referring utterances} in the chain (if any). 


%Speaker BLEU Rogue BERTScore NLG metrics
%Speaker generated utterances- accuracy on listeners  


%------------------------------------------
\subsection{Listener}
\label{sec:listener}
%------------------------------------------

We base our listener model on the reference resolution model proposed by \citet{takmaz-etal-2020-refer} in which the utterance is encoded via BERT~\citep{devlin-etal-2019-bert}. Instead of BERT embeddings, we use word embeddings trained from scratch to ensure that the listeners are genuinely domain-specific. %control for the domain-specificity of the learned embeddings \san{to ensure that the listener is genuinely domain-specific}. 
We concatenate the word embeddings with the representation of the visual context and apply attention over these multimodal representations to obtain the context vector. %The candidate images are also represented multimodally by taking into account previous utterances referring to them in the game \ece{this would no longer apply if we opt for removing history from the listeners/simulators for the sake of adaptation}. 
The listener then selects the target image by comparing the context vector to the representation of the candidate images via dot-product. We train the models with respect to Cross Entropy Loss with the Adam optimizer and select the best model based on resolution accuracy. 

%------------------------------------------
\subsection{Simulators and Speaker Adaptation}
\label{sec:adapts}
%------------------------------------------

The simulator is meant to reflect the idea that the speaker has of the listener. As such, it is a version of the listener model in terms of its architecture. %as it is trained to predict listener behavior \san{I would be more intuitive: the simulator is meant to reflect the idea that the speaker has of the listener. As such, it has the same architecture of the actual listener, but of course it does not\dots}. \ece{although when we input a single latent, there is no sequence and no need for attention} %, but we still have several learnable linear layers with non-linearities to be trained 
%Inspired by the PPLM method, we develop an adaptation scheme where the internal states of the speaker are adapted on-the-fly to produce utterances that are going to be more understandable to the listener, which would improve task accuracy. The updates are performed on the speaker's latent vectors without actually finetuning the speaker. To supply the speaker with the ability to adapt to the listener, we implement a module called the `simulator' that is embedded inside the speaker. The aim of the simulator is to model listener behavior in the speaker's `mind'. To this end, we train a simulator module that learns to predict the listener's output for a given utterance. 
The main differences lie in the data used for training the simulator. This data consists of (1) latent representations from the speaker as input to the simulator, (2) the prediction distributions of the listeners over the candidate images given the speaker's generated utterances as the target outputs for the simulator. The loss is obtained via comparing the simulator's prediction distribution to that of the listener. The models are optimized with respect to KL divergence loss with the Adam optimizer. We select the best model based on the match between the listener's and the simulator's predictions. 

\noindent\textbf{Adaptation} At inference time, we use the desired target to calculate the loss incurred by the simulator. Since this loss is also a reflection of the actual listener loss, we calculate gradient updates based on it. The updates are performed on the latents corresponding to the last hidden states of the speaker's encoder, which are then utilized to initialize the decoder. In this way, we intervene right before decoding and then unroll the decoder based on the updated latent to generate an adapted utterance. We apply a step size of X in updating the latents and also experiment with the number of updates applied in succession.

%\note{mention PPLM?}
%The listeners can also be noisy, unlike PPLM discriminators. Since the trained listener might make mistakes, in the simulator, we utilize both the ground truth and the listener's predicted target to guide the training.
%\ece{this paragraph will mention the ones we actually use out of the following setups}\textit{The gradient updates are applied X times and they go back X timesteps in time. The updates are performed using the latents from distinct parts of the speaker with interventions happening either before/during/after decoding. The `before' option would update the encoder output, which is utilized to initialize the hidden and cell states of the decoder LSTM. In this way, we intervene right before decoding and then unroll the decoder LSTM based on the updated latent. When we use latents from later stages, we update them one-by-one / unroll fully and then update intermediate latents / only the vocabulary-stage latents / start gradient updates from the end (reverse LSTM direction). In these cases, we feed latents at all timesteps as input to the simulator.}

%simulator-listener student-teacher or distillation (predicting the weights highway network, or regressing to the prediction distribution of the listener)

%'hidden', 'hiddencell', 'hiddenwhistory', 'vocablogits', 'vocabprobs'
%h1, c1 = decoderhid, decoderhid
%play with the initialization of the decoder (encoder output, then unroll the LSTM)


%To measure the success of adaptation, we look at whether the speaker's utterance is sufficient enough to predict the target correctly / approximate the prediction distribution / change in the distribution towards a more correct distribution (images ranked based on similarity to target) / the proportion of CLIP-distilled discriminator words in the adapted sentences / in-domain and out-domain adaptation ...


% !TEX root = ../main.tex

%==========================================
\section{Results}
\label{sec:results}
%==========================================

\note{1 page}\\
\note{Ece + Nico}

\raq{the key results are about the listener; the speaker results can be in the appendix}

\subsection{Speaker}

\subsection{Listeners}

\subsection{Adaptive Speaker}

% !TEX root = ../main.tex

%==========================================
\section{Analysis}
\label{sec:analysis}
%==========================================
Possible quantitative analyses / evaluation
\begin{itemize}
\item \textbf{Task-level }evaluation: does the listeners' reference resolution become more accurate as a result of speaker adaptation? We compare listener accuracy and Mean Reciprocal Rank (MRR) in response to adapted utterances with that obtained for utterances generated by a static model \cite{takmaz-etal-2020-refer}.
\item \textbf{Utterance-level}
	\begin{itemize}
	\item We compute several metrics that are commonly used for Natural Language Generation. We consider three measures based on \emph{n-}gram matching---BLEU-2~\cite{Papineni:2002},
%\footnote{BLEU-2, which is based on bigrams, appears to be more informative than BLEU with longer $n$-grams in dialogue response generation \cite{liu-etal-2016-evaluate}}
ROUGE~\cite{Lin2004}, and CIDEr~\cite{cider}---as well as BERTScore F1~\cite{bert-score}, which instead of exact string matches relies on the semantic similarity between tokens. With these measures, we capture the degree of similarity between generated referring utterances and their human counterparts. We expect adapted utterances to be less similar to the human references due to more partner-specific language use (not just transient conceptual pacts but coordination of the entire semantic space).
automatic gen metrics
	\item We also measure the length of the generated utterances to check if the adaptive model reproduces the reduction trend found in humans \cite{haber2019photobook,hawkins2020continual} and in history-aware generation models \cite{takmaz-etal-2020-refer,hawkins2020characterizing}.
	\item The speaker's adaptation may lead to language that is highly effective for the purposes of the image reference task but which does not sound natural, human-like (e.g., repetitions, "heavily" ungrammatical sentences). We will compare the distribution of POS sequences produced by the models with those produced by humans (e.g., we know that Noun Noun sequences become more frequent as we move along a reference chain). We will also use chunking to identify idiosyncratic expressions used by the models and/or by humans.
	\end{itemize}
\item \textbf{Vocabulary-level}
	\begin{itemize}
	\item We will look at the POS distribution over the vocabulary produced by the models and by humans over the entire PhotoBook dataset, as a function of adaptation and of utterance position in the chain.
	\item A possible adaptation strategy for speaker is to reduce its vocabulary to a few very effective words. We check if this is the case by monitoring the vocabulary size and type-token ratio of the models (adaptive and static) and comparing it to humans'.
	\item Another way to assessing how human-like the models' vocabularies are is to check if they follow the Zipfian distribution that we expect from a natural language. 
	\end{itemize}
\item \textbf{Model-level}: We also analyse the effect that adaptation has on the model's internal representations and on its lexical choices.
	\begin{itemize}
	\item What type of internal representations is it more useful to modify? Does adaptation at different representational levels produce different kinds of linguistic output?
	\item How much should the model's hidden representations be modified to strike a good balance between general fluency and partner-specificity? 
	\item How do the model's vocabulary distributions change as a result of adaptation? Do they become less (more) uniform? Do certain successful vocabulary items become more prominent? Are there certain timesteps where the change is stronger?
	\item What effect does adaptation have on the decision making of the model? We can look at the generation samples obtained for adaptive and static models and monitor their overlap as a function of the speed and "depth" of adaptation.
	\end{itemize}
\end{itemize}

Possible qualitative analysis
\begin{itemize}
\item \textbf{Overextension}. Three main types of semantic relations connect conventional and overextended referents of a word \cite{rescorla1980}:
	\begin{itemize}
	\item categorical: linking objects that are close in a taxonomy (e.g., bed referring to a sofa)
	\item analogical: linking objects with shared perceptual properties (e.g., giraffe referring to a lamp)
	\item predicate-based: linking objects that co-occur frequently (e.g., wheels referring to a bike)
	\end{itemize}
Analogical relations are those that can most be exploited by the speaker given a static listener model. Good visual representations should allow the model to find abstract similarities between taxonomically distant and rarely co-occurring objects. 

\end{itemize}
% !TEX root = ../main.tex

%==========================================
\section{Conclusion}
\label{sec:conclusion}
%==========================================

\note{max 1 column}
...



\section*{Limitations}
EMNLP 2022 requires all submissions to have a section titled ``Limitations'', for discussing the limitations of the paper as a complement to the discussion of strengths in the main text. This section should occur after the conclusion, but before the references. It will not count towards the page limit.  

The discussion of limitations is mandatory. Papers without a limitation section will be desk-rejected without review.
ARR-reviewed papers that did not include ``Limitations'' section in their prior submission, should submit a PDF with such a section together with their EMNLP 2022 submission.

While we are open to different types of limitations, just mentioning that a set of results have been shown for English only probably does not reflect what we expect. 
Mentioning that the method works mostly for languages with limited morphology, like English, is a much better alternative.
In addition, limitations such as low scalability to long text, the requirement of large GPU resources, or other things that inspire crucial further investigation are welcome.

\section*{Ethics Statement}
Scientific work published at EMNLP 2022 must comply with the \href{https://www.aclweb.org/portal/content/acl-code-ethics}{ACL Ethics Policy}. We encourage all authors to include an explicit ethics statement on the broader impact of the work, or other ethical considerations after the conclusion but before the references. The ethics statement will not count toward the page limit (8 pages for long, 4 pages for short papers).

%\section*{Acknowledgements}


% Entries for the entire Anthology, followed by custom entries
\bibliography{adapt}
%\bibliography{anthology,custom}
\bibliographystyle{acl_natbib}

\appendix

\section{Appendix}
\label{sec:appendix}

This is a section in the appendix.

\end{document}
